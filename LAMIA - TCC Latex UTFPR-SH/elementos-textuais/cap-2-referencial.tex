\chapter{REVISÃO DA LITERATURA}\label{chp:REVISAO}

O desenvolvimento do trabalho é composto por 3 seções: Revisão da Literatura (ou Referencial Teórico); Metodologia; e Análise dos Resultados, e pode conter outras além dessas. A revisão da literatura deve ser apresentada em forma de texto e seu conteúdo demonstra conhecimento da literatura científica sobre o tema do trabalho. O texto pode ser dividido, para fins didáticos, em subseções. Esta seção é permeada de autores, é o local em que há mais intertextualidade no trabalho. Assim, inclui basicamente citações indiretas (paráfrases) e diretas (curtas e longas). Aqui, o autor explicita a contribuição de outros campos do conhecimento que são envolvidos na pesquisa e outras pesquisas relacionadas ao tema, as conclusões que esses autores chegaram, o que é consenso, as discordâncias entre autores. 


\section{INTERTEXTUALIDADE}\label{sec:INTERTEXTUALIDADE}
No referencial teórico e em outras seções em que a intertextualidade é necessária, devem-se citar trabalhos clássicos, mas priorizar trabalhos dos últimos 10 anos. Podem-se usar artigos científicos, livros, TCCs, dissertações, teses, monografias e sites oficiais. Não são permitidos textos jornalísticos, Wikipédia e de blogues. É importante a utilização de referências em inglês no trabalho, livros e principalmente artigos de revista. O banco do IEEE é uma boa sugestão de fonte de pesquisa nessa língua. 

Devem-se seguir as normas da Associação Brasileira de Normas Técnicas (ABNT)  NBR 10520, Informação e documentação – Citações em documentos – Apresentação,  para fazer a intertextualidade por referenciação. 

\section{ESTADO DA ARTE}\label{sec:ESTADOARTE}

No referencial teórico deve haver uma subseção para o estado da arte, em que se apresenta uma busca de anterioridade sobre o produto a ser desenvolvido, por exemplo, um software ou hardware, uma metodologia, bem como as publicações mais atuais e conceituadas sobre o tema do TCC. Assim, nessa seção, são contextualizados trabalhos anteriores parecidos ou relacionados ao aqui descrito. 

\section{NUMERAÇÃO DAS SEÇÕES}\label{sec:NUMERAÇÃO}
Seguir a ABNT NBR 6024, Informação e documentação – Numeração progressiva das seções de um documento – Apresentação. As seções são formatadas como segue, e podem ir somente até a quaternária:

\begin{table}
	\caption{Numeração progressiva de seção e sua formatação, segundo a ABNT.}
	\label{tab:numeracaosecao}
	\begin{tabular}{|p{14.7cm}|} 
		\hline
		\textbf{\large 1 TÍTULO NÍVEL 1 OU SEÇÃO PRIMÁRIA OU TÍTULO DE CAPÍTULO (TODAS AS LETRAS DE CADA PALAVRA MAIÚSCULA, NEGRITO)}  \\ 
		\hline
		1.1 TÍTULO NÍVEL 2 OU SEÇÃO SECUNDÁRIA (TODAS AS LETRAS DE CADA PALAVRA MAIÚSCULA)~ ~                                   \\ 
		\hline
		1.1.1 Titulo Nível 3 ou Seção Terciária (Primeira Letra de Cada Palavra Maiúscula)~ ~                                   \\ 
		\hline
		1.1.1.1 Titulo nível 4 ou seção quaternária (somente letra da primeira palavra maiúscula)~ ~ ~ ~ ~ ~ ~~                 \\
		\hline
	\end{tabular}
	\newline \footnotesize \textbf{Fonte: Baseado em \cite{Nbr2012}.} 
\end{table}

\section{EQUAÇÕES E ALGORITMOS COM LATEX}\label{sec:LATEX}

\subsection{Equações}\label{sec:Equacoes}
Referência: \url{http://en.wikibooks.org/wiki/LaTeX/Mathematics}

Também: \url{http://en.wikibooks.org/wiki/LaTeX/Advanced_Mathematics}

\begin{equation}
(x + y)^2 = x^2 + 2xy + y^2
\label{eq:Teorema1}
\end{equation}

Referência: \url{https://en.wikipedia.org/wiki/ID3_algorithm}

\begin{equation} \label{eq:DT3} 
\phi^{entropia}(X, y) = -\sum_{l=1}^{k} rac_{\bullet, yl} \times \log_{2} rac_{\bullet, yl}
\end{equation}

\subsection{Algoritmos}\label{sec:Algoritmos}
Referência: \url{http://en.wikibooks.org/wiki/LaTeX/Source_Code_Listings}

\codec{C}{alg:LABEL_CODE_1}{elementos-textuais/codigo-c.txt}

\codejava{Java}{alg:LABEL_CODE_2}{elementos-textuais/codigo-java.txt}

Referência \url{https://www.geeksforgeeks.org/genetic-algorithms/}

\begin{algorithm}
	\caption{Algoritmo Genetico:}
	\label{alg:algoritmogenetico}
	\begin{algorithmic}[1]
		\STATE $d \leftarrow$ Valor como critério de parada
		\STATE $IniciaPopulacao(P, t)$
		\STATE $Avaliacao(P, t)$
		\WHILE{$t < d$}
		\STATE $t \leftarrow t + 1$
		\STATE $SelecionaReprodutores(P, t)$
		\STATE $CruzaSelecionados(P, t)$
		\STATE $MutaResultantes(P, t)$
		\STATE $AvaliaResultantes(P, t)$
		\STATE $AtualizaPopulacao(P, t)$
		\ENDWHILE
	\end{algorithmic}
\end{algorithm}

