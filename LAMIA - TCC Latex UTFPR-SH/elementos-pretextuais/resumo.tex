% SOBRENOME, Prenome do Autor. Título do trabalho: subtítulo. Ano de defesa. 50f. (total de folhas). Trabalho de Conclusão de Curso (Bacharelado em Ciência da Computação) – Universidade Tecnológica Federal do Paraná. Santa Helena. 

Elemento obrigatório, constituído de uma sequência de frases concisas e objetivas, fornecendo uma visão rápida e clara do conteúdo do estudo. O texto deverá conter no máximo 500 palavras e ser antecedido pela referência do estudo. Também, não deve conter citações. O resumo deve ser redigido em parágrafo único, espaçamento simples e seguido das palavras representativas do conteúdo do estudo, isto é, palavras-chave, em número de três a cinco, separadas entre si por ponto e finalizadas também por ponto. Usar o verbo na terceira pessoa do singular, com linguagem impessoal (pronome SE), bem como fazer uso, preferencialmente, da voz ativa.

Para definir as palavras-chave (e suas correspondentes em inglês no abstract) consultar em Termo tópico do Catálogo de Autoridades da Biblioteca Nacional, disponível em: \url{http://acervo.bn.br/sophia_web/index.html} [avaliar se essa informação procede para Ciência da Computação]
